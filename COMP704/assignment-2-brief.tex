\RequirePackage{pdfmanagement-testphase}
\DeclareDocumentMetadata{
	pdfversion=1.7,
}
\documentclass{../common/fal_assignment}
\graphicspath{ {../common/images/} }

\usepackage{enumitem}
\setlist{nosep} % Make enumerate / itemize lists more closely spaced
\usepackage[T1]{fontenc} % http://tex.stackexchange.com/a/17858
\usepackage{url}
\usepackage{todonotes}
\usepackage{float}

% Make it clear this is a draft
\usepackage{draftwatermark}

\title{Development Journal}
%\modulecode{COMP704}
\module{COMP704}
\moduletitle{Machine Learning}

% if moduleinfo is included before setting author, can be overriden :)
\author{Joseph Walton-Rivers}

\version{0.1}
%\programme{MSc Artificial Intelligence for Games}

\begin{document}

\maketitle

\section*{Introduction}

%\begin{marginquote}
%"It isn't enough to pick a path—you must go down it. By doing so, you see things you couldn't possibly see when you started out; you may not like what you see, some of it may be confusing, but at least you will have, as we like to say, 'explored the neighborhood.' The key point here is that even if you decide you're in the wrong place, there is still time to head toward the right place."
%-- Ed Catmull, Creativity Inc.
%
%\end{marginquote}
%\marginpicture{flavour_pic}{
%    \emph{Marvel's Spider-man: PS4}, early gameplay prototype showing web traversal mechanics.
%}

This assignment partners the `Application of Machine Learning' in COMP704 and provides a mechanism to record and journal your experiences, issues and decision making while researching Machine Learning for game AI.

Development journals are often mistaken for personal blogs and it is important that you do not treat this journal like a blog. The journal will be assessed against the marking rubric (below) on the following criteria:

\section*{Problems: description \& scope}
Fundamentally, a development journal is about problems and their solutions. For the problem section, it's good to start with a description of what the nature of the problem is and why it's a problem. 5Ws\&H can be a useful framework for breaking problems down into constituent parts. In addition, a clear understanding of the scope of a problem will be crucial in determining the effort required to address the issue.

\section*{Solutions: synthesis \& quality}
For the solution section of the journal, you should look to develop a narrative as to how the problem was solved or addressed. Often in research, a problem may require several attempts at solution before a robust solution is found. Documenting this journey is a key research activity, so do not hide your failures or partially working solutions as they will often lead you to the solutions you want.

It's also incredibly valuable to write about the quality of solutions that you develop; is a solution merely an interim until something better comes along or does it mark the end of a route of investigation.

\section*{Writing and diagramming}
Writing and diagramming describes the presentation of your journal. Ideally, your writing should communicate your ideas in a style that can be easily understood and followed by others. As a development journal and not a blog, your language should be `professional' rather than colloquial.

Diagramming covers non-textual presentation and usually refers to UML or other visual information models. Ideally, your diagramming, charts and tables should be clear and easy to follow. They should also be referenced from the text that relates to them.

\section*{Further enquiries}
`Further enquiries' describes any follow-on activities that can be taken from the investigations into a particular problem/solution. Often these enquiries may suggest a route that is unrelated to your research and it's useful to mark that in a development journal so that you have the opportunity to come back to it in time. Sometimes, a further enquiry may point your activities into a whole new direction that offers significant advantages and benefits over your current approach. In either case, this information should be recorded.

\section*{Solution reflections}
Finally, for any activities there is significant personal benefit from being a reflective practitioner to think about the problem and solution at hand as well as your approach to addressing the problem. Are there different steps you could apply in the future, did the final solution give you the benefits you wanted and so on?

\section*{Assignment Parts}
The development journal, therefore, goes hand-in-hand with your research and development activity and serves as a record of how and why you did your research alongside your development artefacts that show what you did.

The assignment consists of the following parts:

\subsection*{Part A}
Present an outline research plan. This is a single formative submission. To complete this part of the assignment, attend the workshop in week 4 and give a short presentation which outlines your research plan. You should include material covering what game you are planning to use as a testbed and why, the key AI that you are looking to capture with ML and an outline to the data and ML processing that you are looking to use.

You should use your development journal to record the process of choosing which game you want to work with and why and preparing though initial thoughts for the machine learning process and experiments.

You will receive informal feedback during the session.

\subsection*{Part B}
Attend weekly research progress meetings. This is a single formative submission that is undertaken on a weekly basis during the workshop sessions, where you will be able to discuss your research progress with your peers and with the lecturing staff.

You should use your development journal to record the activities you undertake during your self-directed research. Having your journal present for the weekly research group progress meetings will serve as both a memory aid and allow you to capture useful feedback during the sessions.

You will receive informal feedback during the session.

\subsection*{Part C}
Attend the peer review, this is a single formative submission that will occur after reading week. To complete this part of the assignment, provide a link to your journal before the  peer review session. During the peer review, you can review the work of your colleagues and give suitable support and suggestions for their work.

You will receive informal feedback during the session and formal peer review feedback at the end of the session.

\section*{Part D}
Complete the journal on the Falmouth journal system.
This is a single summative submission. To complete this part, add a link to your development journal to learning space. Your submission will be assessed against the rubric at the end of this document.

You will receive formal feedback within 3 weeks

\section*{Additional Guidance}
This assignment partners the other assignment (Application of Machine Learning) for COMP704 and is designed to be used as a repository for your thoughts recording your experimental outcomes as you develop your data-driven approaches for AI. Do not leave writing the development journal until the end of the assignment as that will greatly diminish its value, and therefore, its mark.

When writing academic journals, many students regard them somewhat like a social media blog, choosing to write trite and short entries, do not make this mistake as this will carry little to no academic value for your journal. Instead, the journal should become your repository for the problems you encounter, the potential solutions you consider and the outcomes of your experimental processes. As ever, the rubric will give you a very clear insight into the expected contents of the journal.

It is also worth drawing on your earlier experiences with reflective practice to create suitable frameworks to consider your experimental activities and the insights that you have gained from them, whether desirable, intended or otherwise.

Finally, don't forget that the journal is an ideal tool for capturing dialogue between yourself and your tutor and over students. Make use of your tutor and the tutorial sessions.

\section*{FAQ}

\begin{itemize}
	\item 	\textbf{What is the deadline for this assignment?} \\ 
    		Falmouth University policy states that summative deadlines must only be specified on the MyFalmouth system.
    		
	\item 	\textbf{What should I do to seek help?} \\ 
    		You can email your tutor for informal clarifications.
    		
    	\item 	\textbf{Is this a mistake?} \\ 	
    		If you have discovered an issue with the brief itself, please inform the module tutor.
    		
\end{itemize}

\rubricmasters
\rubrichead{\ }
\begin{markingrubric}
%
    \firstcriterion{Scope of problems addressed}{15\%}
        \grade\fail Scope is generally poorly defined.
        \grade Problem scope is generally, but not always, clear and unambiguous
        \grade Scope of problems is clear and unambiguous
        \grade Scope is clearly defined and highlights very poignant issues
%
    \criterion{problem/issue description}{15\%}
        \grade \fail Problems presented are often irrelevant to project work.
            \par Often difficult to follow the description of the problem
        \grade Problems are generally relevant
            \par Problems fairly well described
        \grade Relevant problems
            \par Problem generally well described
        \grade Very poignant problems
            \par Descriptions are good or very good
%
    \criterion{Reflections of Solutions}{10\%}
        \grade \fail Generally confusing insights with some occasional clarity
        \grade Generally clear and good quality of reflection but with some unnecessary / confusing insights
            \par Clear and systematic use of a reflective practice framework
        \grade Clear and good quality of reflection
            \par Clear and systematic use of a reflective practice framework
        \grade Very poignant and insightful reflections
            \par Clear and systematic use of a reflective practice framework
%
    \criterion{Writing and diagramming}{20\%}
        \grade \fail Technical writing is lacking in detail and hard to follow. Multiple grammar / layout issues.
            \par UML is generally difficult to follow and sparse
        \grade Generally good technical writing, but with clear grammar / sense issues.
            \par UML is generally clear but with small concept issues
        \grade Good technical writing style with few grammatical issues.
            \par UML diagramming is clear, unambiguous and relevant to writing
        \grade Very clear and understandable writing with no obvious errors or issues.
            \par Good standard of UML that captures all points and aspect required
%
    \criterion{Synthesis of solutions}{15\%}
        \grade \fail Generally confusing path from problems to solutions
        \grade Generally clear train of thought with only the occasional ‘leaps of faith’ or random jumps.
            \par Some evidence of academic referencing.
        \grade Clear and unambiguous train of thought from problem to a workable solution
            \par Clear evidence of relevant academic referencing.
        \grade Exceptionally clear and inspired solutions, backed up with suitable \& highly relevant academic referencing.
%
    \criterion{Quality of solutions}{15\%}
        \grade \fail Solutions may only work in limited conditions / situations and show significant issues in terms of performance, maintainability and/or cost
        \grade Solutions are generally good, but may show minor issues in terms of performance, maintainability and/or cost
        \grade Solutions fit well within programming architecture and are efficient, effective and economical
        \grade Exceptional solutions that show clear novelty in domain application
%
    \criterion{Further Enquiries}{10\%}
        \grade \fail Next steps considered, but not engaged with
        \grade Next steps engaged with in as part of the project
            \par No real consideration given to academic publishing
        \grade Significant consideration given to potential opportunities, issues and other areas for investigation whilst keeping the goal of the project at heart
            \par Some potential for academic publishing opportunities
        \grade Next steps formed a core part of development
            \par Strong potential for academic publishing opportunities
\end{markingrubric}

\end{document}
