\documentclass{../../fal_assignment_opendyslexic}
\graphicspath{ {../../} }

\usepackage{enumitem}
\setlist{nosep} % Make enumerate / itemize lists more closely spaced
\usepackage[T1]{fontenc} % http://tex.stackexchange.com/a/17858
\usepackage{url}
\usepackage{todonotes}
\usepackage{float}

\title{Personal Case Study}
\author{Matt Watkins}
\module{GAM701}
\version{1.0}

\begin{document}

\maketitle

\section*{Introduction}

\begin{marginquote}
“We do not learn from experience; we learn from reflecting on experience.”

— John Dewey

“The reflective practitioner allows himself to experience surprise, puzzlement, or confusion in a situation which he finds uncertain or unique. He reflects on the phenomenon before him, and on the prior understandings which have been implicit in his behaviour. He carries out an experiment which serves to generate both a new understanding of the phenomenon and a change in the situation.”

— Donald Schon
\end{marginquote}
\marginpicture{emissaries}{
	\emph{Ian Cheng's Emissaries} - An example of AI being used in a contemplative artwork.}

The key themes of this module are \textbf{practice} and \textbf{reflection}. No matter which direction, creatively or technically, you choose to pursue, you should aspire to become a reflective practitioner. Such effort will not only empower you to move forward as the creative and technical landscape changes, but also to discover and pioneer new opportunities.

This is a chance for you to develop synergies between your practice and research, and in so doing, gain new and valuable insights into your personal, interpersonal, cultural and technical perspective.

You will analyse the challenges you face through a series of \textbf{investigations}. These investigations will allow you to reflect on the future possibilities of your practice and professional development. You will then distill your thoughts and processes into a series of submissions.

This assignment is formed of three main parts:

\begin{itemize}
		\item (1)  \textbf{Write} a  \textbf{report} on continuing personal and professional development that focuses and reflects upon:
		\begin{itemize}
			\item a.  \textbf{key skills} that define your practice;
			\item b.  \textbf{challenges} and limitations you have found in your skillset;
			\item c.  \textbf{foresee} tangible plans to improve and overcome your identified limitations.
		\end{itemize}
		\item (2)  \textbf{Maintain} a \textbf{Critical Reflective Journal} (CRJ) throughout the module in which you:
		\begin{itemize}
			\item a.  \textbf{reflect} on your work on a weekly basis;
			\item b.  \textbf{keep track} of your learning, challenges and further lines of interest.
		\end{itemize}
		\item (3)  \textbf{Record} a  \textbf{short video} that:
		\begin{itemize}
			\item a.  \textbf{summarises} your experience as a reflective practitioner;
			\item b.  \textbf{highlights} the main challenges, and ways to overcome them in the future.
		\end{itemize}
\end{itemize}
\subsection*{Assignment Setup} 

This assignment embeds your insights as a reflective practitioner throughout the module, as well as potentially drawing bridges with other modules or topics you find relevant to your practice.

Submission must be done as a single compressed ZIP file containing the re- quired items for each part of the assignment.
\pagebreak
\section*{Submissions}
\subsection*{Part 1 - Critical Reflective Journal} 

Critical reflective writing is an opportunity for self-exploration, and an opportunity for you to attain self-understanding through the analysis of your personal characteristics and experiences. You are expected to write a \textbf{blog entry each week} where you reflect on your creative practice.

Your journal is a place for you to:
\begin{itemize}
	\item (a) Practise writing in analytical and evaluative styles rather than a descriptive style;
	\item (b) Analyse the concepts you develop through your ongoing deliberate practice;
	\item (c) Examine how you’ve applied your research and directed learning to your practice-based projects;
	\item (d) Look backward to see what you have accomplished so far;
	\item (e) Project forward to desirable goals you might obtain in the future.
\end{itemize}

To complete this assignment, you are expected to set up a blog (using either \textbf{Wordpress}, or \textbf{GitHub Pages}) and complete a short journal entry each week, from Week 1 up to the submission deadline.

\emph{Note - You should choose \textbf{only one} of the following options to complete.}
	
\subsection*{Part 2 - Report (Option 1)} 

To complete this option of the assignment you will critically reflect on your learning and pathway specific practice across the whole study block. This involves identifying the key factors that influenced the quality of your work and then, from this, develop a plan for continuing professional development.

Such reflection and planning is an extremely important part of developing skills as creative digital designers and developers. Research shows that deliberate practice is very effective at nurturing expertise in creative practice.

Write a \textbf{2700-word report} to:
\begin{itemize}
\item(a) Describe what you have accomplished so far, focusing on key skills required by your profile and specialisms.
\item (b) Briefly justify the relevance and importance of each of these skills, with respect to your professional development
\item(c) Structure your thoughts and reflections, focusing on lessons learned from the workshops and seminar sessions.
\item(d) Outline key areas of practice you aim to develop across the rest of the module.
\item(e) Propose directions for further research and self-exploration that will support such development.
\item (f) Suggest how to overcome each of these challenges or obstacles, with reference to SMART actions.
\end{itemize}

Please submit your work in \textbf{.PDF format}.

You can ask for immediate \textbf{informal feedback} from your \textbf{tutor} during both meetings and workshop sessions.

\emph{Note - The word count is a limit, and not a target. Aim to write up to 500-words for each skill, around 100- words for the introduction, and around 100-words for the conclusion.}

You can ask to receive \textbf{informal feedback} from your \textbf{tutor} at any point through- out the study block.

\emph{Note - you are not required to include holidays, but you are welcome to do so if you choose to engage in creative and/or technical practice during such periods.}


\subsection*{Part 2 - Video (Option 2)} 

To complete this option of the assignment, you are expected to record and edit a video of no more than \textbf{10 minutes}. 

Upload a link to \textbf{YouTube, Vimeo or a file upload service}.

This video should be a reflection space for you to:
\begin{itemize}
\item(a) Describe what you have accomplished so far, focusing on key skills required by your profile and specialisms.
\item (b) Briefly justify the relevance and importance of each of these skills, with respect to your professional development
\item(c) Structure your thoughts and reflections, focusing on lessons learned from the workshops and seminar sessions.
\item(d) Outline key areas of practice you aim to develop across the rest of the module.
\item(e) Propose directions for further research and self-exploration that will support such development.
\item (f) Suggest how to overcome each of these challenges or obstacles, with reference to SMART actions.
\end{itemize}
You will be assessed on your ability to create a coherent and well edited video. It does not need to be of high quality in terms of production, but the story of your reflective journey should be clearly mapped out and articulated.

\section*{Additional Guidance}

Unlike many other modules, in which the learning process is based on assimilating external content, this module aims to boost your learning and skills the other way around; namely, by focusing on your own way of working. Reflective practice is, therefore, the beginning and the end of this module. Whatever the field of expertise you choose to work on, you will be faced with opportunities and challenges that will ask you to think differently about the way you use your tools, and which will require not only to know which approaches to use, but also to understand and justify why those approaches were chosen to tackle a particular problem.

As such, and even though some small-scale artefacts will be created in the numerous workshops, the real goal of this module, and thus the final artefact, is a work of reflection and introspection into your work and practice. Therefore, those small-scale artefacts are not the destination, but rather the journey that will foster your reflection and that will allow you to engage in an analysis of your own tools, techniques and approaches.

Try to reflect holistically by drawing examples of challenges/obstacles from a range of different domains: the affective domain, such as your ability to identify and manage different mental states and emotions; the interpersonal domain, such as your ability to communicate and organise activity with peers; the dispositional domain, such as your ability to manage time effectively and remain disciplined; the cognitive domain, such as your conceptual knowledge of the relevant tools in your specialty; and the procedural domain, such as your abilities to apply techniques and problem solving skills. The more specific and granular the challenges / obstacles you highlight, the easier it will be to define actions to improve. There should be a clear link between the action, and the skill the action aims to improve. Also recall that the term ’SMART action’ refers to a specific, measurable, achievable, relevant, and time-bound action.

\pagebreak
\section*{FAQ}

\begin{itemize}
	\item 	\textbf{What is the deadline for this assignment?} \\ 
    		Falmouth University policy states that summative deadlines must only be specified on the MyFalmouth system.
	\item 	\textbf{What should I do to seek help?} \\ 
    		You can email your tutor for informal clarifications.    		
    	\item 	\textbf{Is this a mistake?} \\ 	
    		If you have discovered an issue with the brief itself, please inform the module tutor.
\end{itemize}

\section*{Additional Resources}
\begin{itemize}
\item Fook, J. et al. 2007. Practising critical reflection: a resource handbook. Open University Press.
\item Ericsson, K.A. et al. 1993. The role of deliberate practice in the acquisition of expert performance. Psychological Review. 100, 3 (1993), 363–406.
\item Boud, D. 2001. Using journal writing to enhance reflective practice. New Directions for Adult and Continuing Education. 2001, 90 (2001).
\item Babb, J. et al. 2014. Embedding Reflection and Learning into Agile Software Development. IEEE Software. 31, 4 (Jul. 2014), 51–57.

DOI:https://doi.org/10.1109/MS.2014.54.
\item Ericsson, K.A. and Credo Reference (Firm) 2006. The Cambridge handbook of expertise and expert performance. Cambridge University Press.
\item Measey, P. and Radtac 2015. Agile Foundations: principles, practices and frameworks. BCS.
\item Principles behind the Agile Manifesto: 

http://agilemanifesto.org/principles.html.
\item Hatch, M. 2014. The maker movement manifesto: rules for innovation in 	the new world of crafters, hackers, and tinkerers. McGraw-Hill Education.
\item Wilkinson, K. et al. 2013. The art of tinkering: meet 150+ makers working at the intersection of art, science \& technology. Weldon Owen.
\end{itemize}

\rubrichead{All submissions and assessment criteria for this assignment are individual.}
\rubricmasters

\begin{markingrubric}
	\firstcriterion{\textbf{INDUSTRY:} \par Evaluate the pertinent legal, social, ethical, and professional issues associated with live development contexts.\par}{30\%}
	\grade\fail The student demonstrates little or no understanding of the legal, social, ethical and professional issues associated with live development contexts
	\grade The student demonstrates some ability to evaluate the legal, social, ethical and professional issues associated with live development contexts, but this is not always consistently applied.
	\grade The student demonstrates ability to critically evaluate the legal, social, ethical and professional issues associated with live development contexts.
	\grade The student demonstrates ability to critically evaluate the legal, social, ethical and professional issues associated with live development contexts, leading to new contextual insight.
	%
	\criterion{\textbf{ANALYSIS:} \par Critically reflect upon and evaluate working methods and solutions.\par}{70\%}
	\grade\fail The student demonstrates little or no ability to critically reflect upon and evaluate working methods and solutions effectively and efficiently.
	\grade The student demonstrates some ability to critically reflect upon and evaluate working methods and solutions effectively and efficiently.
	\grade The student demonstrates ability to reflect upon and evaluate working methods and solutions effectively and efficiently. Critical reflection is communicated confidently in a range of formats and contexts.
	\grade The student demonstrates ability to reflect upon and evaluate working methods and solutions effectively and efficiently. Critical reflection is communicated confidently in a range of formats and contexts, articulating ideas fluently and eloquently.
	%
\end{markingrubric}
\end{document}