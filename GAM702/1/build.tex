\documentclass{../../fal_assignment}
\graphicspath{ {../../} }

\usepackage{enumitem}
\setlist{nosep} % Make enumerate / itemize lists more closely spaced
\usepackage[T1]{fontenc} % http://tex.stackexchange.com/a/17858
\usepackage{url}
\usepackage{todonotes}
\usepackage{float}

\title{Portfolio of Game Prototypes}
\author{Brian McDonald}
\module{GAM702}
\version{1.0}

\begin{document}

\maketitle

\section*{Introduction}

\begin{marginquote}
"It isn't enough to pick a path—you must go down it. By doing so, you see things you couldn't possibly see when you started out; you may not like what you see, some of it may be confusing, but at least you will have, as we like to say, 'explored the neighborhood.' The key point here is that even if you decide you're in the wrong place, there is still time to head toward the right place."
-- Ed Catmull, Creativity Inc.

\end{marginquote}
\marginpicture{flavour_pic}{
    \emph{Marvel's Spider-man: PS4}, early gameplay prototype showing web traversal mechanics.
}

For this assignment you will create \textbf{five} prototype games based on a series of provocations provided by the module tutor. In effect, you will be creating a small game prototype every two weeks. After the end of the two weeks, you will receive feedback from your Peers and the Module Tutors.

As a Game Designer it is up to you on what kind of game you create and what technology you will use to create the prototype. You can create a digital game, boardgame, cardgame, RPG Module or physical game. However, you will need to justify all your decisions in the Development Journal , see \textbf{Assignment 2}

For the final submission you have to select \textbf{three} prototypes that will go forward into your final summative submission.

\subsection*{Peer and self Assessment}
After each formative submission you will be required to play a selection of prototypes produced by your peer and even your own game. It is required that you engage with
this process through out the module. Every piece of feedback should be constructive and actionable, in addition, a portion of your mark will be derived on how you engage with this
feedback process. 

\subsection*{Assignment Setup} 

This assignment consists of \textbf{five formative submissions}, followed by a \textbf{single summative submission}.

After each formative submission you will receive feedback from your peers and module tutor. You should note this feedback and feed this into subsequent prototypes.  

The formative submissions consists of a single zip file, with the following folder structure. You can also find a template zip file on the Assignment Space in the Learning Space

\pagebreak
\subsection*{Digital Game Submission} 

\begin{figure}[H]
	\begin{center}
		\includegraphics[height=0.4\textheight]{digital_games_folder_structure}
	\end{center}
	\caption{Recommended folder structure for formative submissions of Digital Games.}
	\label{fig:digital_game_folder_structure}
\end{figure}

\begin{itemize}
	\item The \textbf{instructions.txt} should contain any controls and other information required to play
	\item The \textbf{readme.txt} should contain a description of the game and any references to resources used for the game. These resources include assets, reference materials, tutorials etc
	\item The \textbf{src} folder should contain the project files and source code for the game
	\item The \textbf{bin} folder should contain project compiled executable
	\item The \textbf{assets} folder should contain all source assets used in the game including images, documents or text
\end{itemize}

\pagebreak
\pagebreak
\subsection*{Physical Game (including Boardgame, RPG and Folk Game)} 

\begin{figure}[H]
	\begin{center}
		\includegraphics[height=0.4\textheight]{physical_games_folder_structure}
	\end{center}
	\caption{Recommended folder structure for formative submissions of Physical Games.}
	\label{fig:phyical_game_folder_structure}
\end{figure}

\begin{itemize}
	\item The \textbf{components.pdf} should contain a list of all components (dice, cubes, coins etc) required to play the game
	\item The \textbf{instructions.pdf} should be in the style of a board game manual, with setup instructions, a how to play guide and detailed rule instructions
	\item The \textbf{images} folder should contain images of the components, game setup and gameplay
	\item The \textbf{assets} folder should contain all source assets used in the game including images, documents or text
\end{itemize}

If you need a good example of rulebook layout please look at the following

\url{https://www.fairway3games.com/writing-rules-a-recipe/}
\url{https://www.orderofgamers.com/downloads/MansionsofMadness2ndEd_v1.pdf}

\pagebreak
\subsection*{Final Submission} 

\begin{figure}[H]
	\begin{center}
		\includegraphics[height=0.4\textheight]{portfolio_folder_structure}
	\end{center}
	\caption{Recommended folder structure for final submission.}
	\label{fig:portfolio_folder_structure}
\end{figure}

At the end of the semester you will be required make a final summative submission of \textbf{three} of your \textbf{five} prototypes. 
Prepare a \textbf{single \texttt{.zip} file} containing your submissions \textbf{see folder structure}, and upload it to the appropriate submission area on LearningSpace.

\textbf{This final submission is subject to the usual university policies regarding late submission or non-submission,
as detailed in the course handbook ---
even if you have met all the formative deadlines,
failure to make a submission via LearningSpace by the summative deadline will be subject to penalties.}

\section*{Additional Guidance}



\section*{FAQ}

\begin{itemize}
	\item 	\textbf{What is the deadline for this assignment?} \\ 
    		Falmouth University policy states that summative deadlines must only be specified on the MyFalmouth system.
    		
	\item 	\textbf{What should I do to seek help?} \\ 
    		You can email your tutor for informal clarifications.  
    		
	\item 	\textbf{How will I receive feedback on my work?} \\ 
    		You will be given verbal feedback on your work during the session in which it is marked.
    		If you require more in-depth feedback or discussion, please book an appointment with your tutor.
    		
    	\item 	\textbf{Is this a mistake?} \\ 	
    		If you have discovered an issue with the brief itself, please inform the module tutor.
\end{itemize}

\rubrichead{All submissions and assessment criteria for this assignment are individual.}
\begin{markingrubric}
	%
	\firstcriterion{Basic Competency Threshold}{40\%}
	\gradespan{1}{\fail At least one part is missing or is inadequate.}
	\gradespan{5}{Adequate ability to generate ideas, problem solving, concepts and proposals in response to set briefs and/or self-initiated activity.
		\par The work demonstrates an adequate, ethically informed, real-world experience of industry/business environments and markets.
		\par Enough work is available to hold a meaningful discussion.
		\par Adequate participation in-class peer-review activities
		\par No breaches of academic integrity.}
	%
	\criterion{Functional Coherence of Code}{10\%}
	\grade\fail 	No algorithm has been implemented successfully.
	\par 		The source code does not compile or there are serious syntax errors.
	\grade 		At least one algorithm has been  implemented successfully.
	\par 		There are many obvious logical errors, more than one of which is significant.   
	\grade 		At least two algorithms have been  implemented successfully.
	\par 		There are several obvious logical errors, no more than one of which is significant. 
	\grade 		At least three algorithms have been implemented successfully.
	\par 		There are some obvious logical errors, which are not significant. 
	\par		The brief has been satisfied.
	\grade 		At least three algorithms have been  implemented successfully.
	\par 		There are few obvious logical errors, which are cosmetic and/or superficial.
	\par		The brief has been satisfied.     
	\grade 		At least three algorithms have been  implemented successfully.
	\par		There are no obvious logical errors.
	\par		The brief has been satisfied.
	%
	\criterion{Sophistication of Code}{20\%}
	\grade\fail No insight into the appropriate use of programming constructs is evident from the source code.
	\par No attempt to structure the program (e.g. one monolithic function).
	\grade Little insight into the appropriate use of programming constructs is evident from the source code.
	\par The program structure is poor.
	\grade Some insight into the appropriate use of programming constructs is evident from the source code.
	\par The program structure is adequate.
	\grade Much insight into the appropriate use of programming constructs is evident from the source code.
	\par The program structure is appropriate.
	\grade Considerable insight into the appropriate use of programming constructs is evident from the source code.
	\par The program structure is effective. There is high cohesion and low coupling.
	\grade Significant insight into the appropriate use of programming constructs is evident from the source code.
	\par The program structure is very effective. There is high cohesion and low coupling.
	%
	\criterion{Novelty}{10\%}
	\grade\fail There are no comments in the source code, or comments are misleading.
	\par Most variable names are unclear or inappropriate.
	\par Code formatting hinders readability.
	\grade The source code is only sporadically commented, or comments are unclear.
	\par Some identifier names are unclear or inappropriate.
	\par Code formatting is inconsistent or does not aid readability.
	\grade The source code is somewhat well commented.
	\par Some identifier names are descriptive and appropriate.
	\par An attempt has been made to adhere to thhe PEP-8 formatting style.
	\par There is little obvious duplication of code or of literal values.           
	\grade The source code is reasonably well commented.
	\par Most identifier names are descriptive and appropriate.
	\par Most code adheres to the PEP-8 formatting style.
	\par There is almost no obvious duplication of code or of literal values.   
	\grade The source code is reasonably well commented, with Python doc-strings.
	\par Almost all identifier names are descriptive and appropriate.
	\par Almost all code adheres to the PEP-8 formatting style.
	\par There is no obvious duplication of code or of literal values. Some literal values can be easily ``tinkered'' in the source code. 
	\grade The source code is very well commented, with Python doc-strings.
	\par All identifier names are descriptive and appropriate.
	\par All source code adheres to the PEP-8 formatting style.
	\par There is no obvious duplication of code or of literal values. Most literal values are, where appropriate, easily ``tinkered'' outside of the source code.  
	%
	\criterion{Creativity of the Prototypes}{20\%}
	\grade\fail No creativity.
	\par The work is a clone of an existing work with mere cosmetic alterations.
	\grade Little creativity.
	\par The work is derivative of existing works, with only minor alterations.
	\grade Some creativity.
	\par The work is derivative of existing works, demonstrating little divergent and/or subversive thinking.
	\grade Much creativity.
	\par The work is somewhat novel, demonstrating some divergent and/or subversive thinking.
	\grade Considerable creativity.
	\par The work is novel, demonstrating significant divergent and/or subversive thinking.
	\grade Significant creativity.
	\par The work is highly original, with strong evidence of divergent and/or subversive thinking.
	%

\end{markingrubric}

\end{document}