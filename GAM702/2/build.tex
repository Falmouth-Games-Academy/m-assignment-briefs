\documentclass{../../fal_assignment}
\graphicspath{ {../../} }

\usepackage{enumitem}
\setlist{nosep} % Make enumerate / itemize lists more closely spaced
\usepackage[T1]{fontenc} % http://tex.stackexchange.com/a/17858
\usepackage{url}
\usepackage{todonotes}
\usepackage{float}

\title{Development Journal}
\author{Brian McDonald}
\module{GAM702}
\programme{MA Game Design}
\version{1.0}

\begin{document}

\maketitle

\section*{Introduction}

\begin{marginquote}
"It isn't enough to pick a path—you must go down it. By doing so, you see things you couldn't possibly see when you started out; you may not like what you see, some of it may be confusing, but at least you will have, as we like to say, 'explored the neighborhood.' The key point here is that even if you decide you're in the wrong place, there is still time to head toward the right place."
-- Ed Catmull, Creativity Inc.

\end{marginquote}
\marginpicture{flavour_pic}{
    \emph{Marvel's Spider-man: PS4}, early gameplay prototype showing web traversal mechanics.
}

For this assignment you will create and maintain a blog that will detail your development process during the course of the module. The goal of the blog is to capture your development process and the lessons you have learned from each prototype. This blog is not simply a review of each game, but a critical evaluation of your process and practices.

You should make a minimum of \textbf{three posts per game}.

\begin{itemize}
	\item First post which details your initial thoughts about the brief and provides initial brainstorming
	\item Second post, which would usually be mid-way through development which updates the reader on the progress so far
	\item Third post, this reflects on the development of the prototype and details what you have learned so far
\end{itemize}

Before the final submission you should also make a post which contains the following

\begin{itemize}
	\item A rationale of why you selected the 3 prototypes for submission
	\item What lessons you have learned from the development of the 5 prototypes
	\item Reflect on what you have learned as Designer
	\item State how you would approach a hypothetical 6th prototype 
\end{itemize}

For the final submission you should provide a link to your blog, to the submission area of the assignment on the Learning Space. 

\subsection*{Assignment Setup} 

To setup your blog, you should visit the following referral link \url{https://journal.falmouth.ac.uk/?join-invite-code=990-gam702}. This will guide you
through the creation of the blog and add you to the class list. Please ensure that you name the blog something sensible and you pick a theme which makes the blog clear and readable.

This assignment consists of \textbf{five formative submissions}, followed by a \textbf{single summative submission}.

After each formative submission you will receive feedback from your peers and module tutor. You should note this feedback and feed this into subsequent blog posts.  

The formative submissions consists of adding a link to the blog on the submission area on the 


\section*{Additional Guidance}



\section*{FAQ}

\begin{itemize}
	\item 	\textbf{What is the deadline for this assignment?} \\ 
    		Falmouth University policy states that summative deadlines must only be specified on the MyFalmouth system.
    		
	\item 	\textbf{What should I do to seek help?} \\ 
    		You can email your tutor for informal clarifications.  
    		
	\item 	\textbf{How will I receive feedback on my work?} \\ 
    		You will be given verbal feedback on your work during the session in which it is marked.
    		If you require more in-depth feedback or discussion, please book an appointment with your tutor.
    		
    	\item 	\textbf{Is this a mistake?} \\ 	
    		If you have discovered an issue with the brief itself, please inform the module tutor.
\end{itemize}

\rubrichead{All submissions and assessment criteria for this assignment are individual.}
\begin{markingrubric}
	%
	\firstcriterion{Basic Competency Threshold}{40\%}
	\gradespan{1}{\fail At least one part is missing or is inadequate.}
	\gradespan{5}{Adequate ability to generate ideas, problem solving, concepts and proposals in response to set briefs and/or self-initiated activity.
		\par The work demonstrates an adequate, ethically informed, real-world experience of industry/business environments and markets.
		\par Enough work is available to hold a meaningful discussion.
		\par Adequate participation in-class peer-review activities
		\par No breaches of academic integrity.}
	%
	\criterion{Appropriateness of tools and techniques}{15\%}
	\grade\fail 	No appropriate tools or techniques are used
	\par 		The tools selected through out the project have not matched the challenge of the briefs.
	\grade 		Only one tool/technique has been used throughout the project.
	\par 		The knowledge of this tool is fairly basic.  
	\grade 		Only one tool/technique has been used throughout the project.
	\par 		The knowledge of this tool has advanced throughout the project.
	\grade 		There has been a mix of two to three tools or techniques used. 
	\par 		The knowledge of these tools/techniques are fairly basic. 
	\par		The tools/techniques selected feel justified in terms of the game idea
	\grade 		There has been a mix of two to three tools or techniques used.
	\par 		The knowledge of this tool has advanced throughout the project.
	\par		The briefs have been satisfied.     
	\grade 		There has been a mix of two to three tools or techniques used
	\par		The knowledge exhibited of these tools are at near expert level
	\par		The briefs has been satisfied.
	%
	\criterion{Evolution of practice}{15\%}
	\grade\fail There has no evolution of practice.
	\par All games feel similar.
	\grade There is little evolution of practice.
	\par It feels like the designer hasn't properly learned from past prototypes.
	\grade This is some evolution of practice.
	\par There are some links in terms of lessons learned from previous projects.
	\grade There is much evolution of practice.
	\par There are visible direct links in terms of lessons learned from previous projects.
	\grade Considerable evolution of practice.
	\par Each prototype feels like a piece in a body of work.
	\grade Significant evolution of practice.
	\par There are clear lessons learned from each project.
	\par Each prototype feels like a piece in a body of work.
	%
	\criterion{Peer Assessment}{10\%}
	\grade\fail No engagement.
	\par The feedback was non existent or not useful at all
	\grade Little engagement.
	\par The feedback is not constructive or actionable.
	\grade Some engagement.
	\par The feedback is useful but doesn't go into enough detail to make any real impact.
	\grade Much engagement.
	\par The feedback has actionable points but requires more detail to be truly useful
	\grade Considerable engagement.
	\par The feedback is excellent, and contains many actionable points.
	\grade Significant engagement.
	\par The feedback is exemplary, it is actionable, contains alternative suggestions and approaches.
	%
	\criterion{Creativity of the Prototypes}{20\%}
	\grade\fail No creativity.
	\par The work is a clone of an existing work with mere cosmetic alterations.
	\grade Little creativity.
	\par The work is derivative of existing works, with only minor alterations.
	\grade Some creativity.
	\par The work is derivative of existing works, demonstrating little divergent and/or subversive thinking.
	\grade Much creativity.
	\par The work is somewhat novel, demonstrating some divergent and/or subversive thinking.
	\grade Considerable creativity.
	\par The work is novel, demonstrating significant divergent and/or subversive thinking.
	\grade Significant creativity.
	\par The work is highly original, with strong evidence of divergent and/or subversive thinking.
	%

\end{markingrubric}

\end{document}