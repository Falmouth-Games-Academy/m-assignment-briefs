\documentclass{../../fal_assignment}
\graphicspath{ {../../} }

\usepackage{enumitem}
\setlist{nosep} % Make enumerate / itemize lists more closely spaced
\usepackage[T1]{fontenc} % http://tex.stackexchange.com/a/17858
\usepackage{url}
\usepackage{todonotes}
\usepackage{float}

\title{Development Journal}
\author{Brian McDonald}
\module{GAM702}
\programme{MA Game Design}
\version{1.0}

\begin{document}

\maketitle

\section*{Introduction}

\begin{marginquote}
"It isn't enough to pick a path—you must go down it. By doing so, you see things you couldn't possibly see when you started out; you may not like what you see, some of it may be confusing, but at least you will have, as we like to say, 'explored the neighborhood.' The key point here is that even if you decide you're in the wrong place, there is still time to head toward the right place."
-- Ed Catmull, Creativity Inc.

\end{marginquote}
\marginpicture{flavour_pic}{
    \emph{Marvel's Spider-man: PS4}, early gameplay prototype showing web traversal mechanics.
}

For this assignment you will create and maintain a blog that will detail your development process during the course of the module. The goal of the blog is to capture your development process and the lessons you have learned from each prototype. This blog is not simply a review of each game, but a critical evaluation of your process and practices.

You should make a minimum of \textbf{three posts per game}.

\begin{itemize}
	\item First post which details your initial thoughts about the brief and provides initial brainstorming
	\item Second post, which would usually be mid-way through development which updates the reader on the progress so far
	\item Third post, this reflects on the development of the prototype and details what you have learned so far
\end{itemize}

Before the final submission you should also make a post which contains the following

\begin{itemize}
	\item A rationale of why you selected the 3 prototypes for submission
	\item What lessons you have learned from the development of the 5 prototypes
	\item Reflect on what you have learned as Designer
	\item State how you would approach a hypothetical 6th prototype 
\end{itemize}

For the final submission you should provide a link to your blog, to the submission area of the assignment on the Learning Space. 

\subsection*{Assignment Setup} 

To setup your blog, you should visit the following referral link \url{https://journal.falmouth.ac.uk/?join-invite-code=990-gam702}. This will guide you
through the creation of the blog and add you to the class list. Please ensure that you name the blog something sensible and you pick a theme which makes the blog clear and readable.

In addition, please upload your a URL of your blog to the LearningSpace to the \textbf{StudentBlog} area.

This assignment consists of \textbf{five formative submissions}, followed by a \textbf{single summative submission}.

After each formative submission you will receive feedback from your peers and module tutor. You should note this feedback and feed this into subsequent blog posts.  

The formative submissions consists of adding a link to the blog on the submission area on the Learning Space.


\section*{Additional Guidance}

It is important to keep the blog up to date, you should try to get into the habit of updating your blog, make it part of your practice.   

For some good advice, please look at these resources:
\begin{itemize}
	\item \url{http://thegameshub.com/how-to-write-an-amazing-indie-gamedev-blog/}
	\item \url{https://blog.hubspot.com/marketing/8-essential-writing-tips}
	\item \url{https://wpnewsify.com/blog/write-perfect-wordpress-blog-post/}
\end{itemize}

For some good examples, please check out the following:
\begin{itemize}
	\item \url{https://www.positech.co.uk/cliffsblog/}
	\item \url{https://grumpygamer.com/}
	\item \url{https://www.pentadact.com/category/making-games/}
\end{itemize}


\section*{FAQ}

\begin{itemize}
	\item 	\textbf{What is the deadline for this assignment?} \\ 
    		Falmouth University policy states that summative deadlines must only be specified on the MyFalmouth system.
    		
	\item 	\textbf{What should I do to seek help?} \\ 
    		You can email your tutor for informal clarifications.  
    		
	\item 	\textbf{How will I receive feedback on my work?} \\ 
    		You will be given verbal feedback on your work during the session in which it is marked.
    		If you require more in-depth feedback or discussion, please book an appointment with your tutor.
    		
    	\item 	\textbf{Is this a mistake?} \\ 	
    		If you have discovered an issue with the brief itself, please inform the module tutor.
\end{itemize}

\begin{markingrubric}
	%
	\firstcriterion{Basic Competency Threshold}{40\%}
	\gradespan{1}{\fail Did not post the minimum posts per prototype}
	\gradespan{5}{The student posted 3 blog posts per prototype.
		\par The student posted a final blog post.
		\par The standard of English was a decent standard.
	}
	%
	\criterion{Standard of blog posts}{20\%}
	\grade\fail	The blog post are very minimal.
	\par		There are only a few lines of text.
	\grade	    The Blog posts has a fair structure with supporting references
	\grade		The Blog posts has a good structure with supporting references
	\par		The posts have images of the game in development
	\grade		The Blogs have a very good structure with support references
	\par			The posts have images of the game in development
	\grade		The Blogs have an excellent structure with support references
	\par			The posts have images of the game in development
	\grade		The Blogs are of an industry standard
	%
	\criterion{Reflection}{15\%}
	\grade\fail These is no reflection on the prototypes or practice
	\grade		Little insight is demonstrated
	\par		The student has carried out minimal reflection on their practice
	\grade		Some insight is demonstrated
	\par		The student has carried out some reflection on their practice
	\grade		Much insight is demonstrated
	\par		The student has carried out a good level reflection on their practice
	\par        There is evidence of this reflection feeding forward into the next blog post
	\grade		Considerable insight is demonstrated
	\par		The student has carried out a very good level reflection on their practice
	\par         There is a narrative flowing through each blog post 
	\grade		Significant insight is demonstrated.
	\par		The student has carried out an excellent level reflection on their practice
	\par 		Each blog post builds on the previous one and demonstrates a real growth as a designer
	%
	\criterion{Synthesis}{10\%}
	\grade\fail No attempt has been made to synthesise information from multiple sources.
	\grade		A superficial attempt has been made to synthesise information from multiple sources.
	\grade		A reasonable attempt has been made to synthesise information from multiple sources.
	\grade		Information from multiple sources is synthesised into a somewhat cohesive whole.
	\grade		Information from multiple sources is synthesised into a cohesive whole.
	\par		Connections are analytical in nature.
	\grade		Information from multiple sources is synthesised into a strongly cohesive whole.
	\par		Connections are analytical and evaluative in nature.
	%
	\criterion{Spelling \& grammar}{5\%}
	\grade\fail 	Substantial spelling and/or grammatical errors.
	\grade 		Many spelling and/or grammatical errors.
	\grade 		Some spelling and/or grammatical errors.  
	\grade 		Few spelling and/or grammatical errors.
	\grade 		Almost no spelling and/or grammatical errors.
	\grade 		No spelling or grammatical errors.
	%
	\criterion{Structure}{10\%}
	\grade\fail 	There is no structure, or the structure is unclear.
	\grade 		There is little structure.
	\grade 		There is some structure.
	\par 		A few sentences and paragraphs are well constructed.
	\grade 		There is much structure.
	\par 		Some sentences and paragraphs are well constructed.
	\grade 		There is much structure, highlighting the key themes.
	\par 		Most sentences and paragraphs are well constructed.
	\grade 		There is much structure, highlighting the key themes.
	\par 		All sentences and paragraphs are well constructed.
\end{markingrubric}


\end{document}