\chapter{Introduction}

You are required to deliver a \textbf{major project} as part of your Masters degree; in the form of \textbf{research and/or innovation} relating to your specialism, encompassing \textbf{empirical} and/or \textbf{practice-based} research. Individually, you will explore a field that interests you, and for which there is a clearly identified need.

The higher level aim for this module is to support you on your journey through a research and innovation project that has the potential to contribute new knowledge in a chosen field of study, produce an artefact that is of publishable quality, and develop your personal portfolio in a way that supports your future aspirations. If your project is particularly strong, there may be scope for taking it forwards for publication in academic conferences, or seeking business start-up funding.

\section*{Project scope}

The brief for this project is intentionally open-ended. You will be assigned a \textbf{supervisor}, a member of academic staff whose role is to mentor you through the project. With your supervisor's aid, you will refine your proposal and then deliver your project.

Your project may be:

\begin{itemize}
    \item \textbf{A standalone individual project}, where you undertake all of the work yourself.
    \item \textbf{An individual contribution to a group project with other GAM705 students}. The maximum size of a group in this case is 4, and each member of the group must have a well-defined contribution to the overall project.
    \item \textbf{An individual contribution to an external project}. Some students at this stage of their studies may be working as interns or part-time employees in industry, as members of teams under incubation schemes such as Transfuzer, or collaborating with students from other courses. In this case, you may leverage this work as the basis for your major project
    as long as:
    \begin{itemize}
        \item you are able to fulfil the requirements of this assignment;
        \item you are making a well-defined contribution to the external project; and
        \item you obtain permission from your external partners, in particular addressing any issues around intellectual property or commercially sensistive information.
    \end{itemize}
\end{itemize}

The main aim for this project is to produce some kind of \textbf{artefact}. Again, the definition of ``artefact'' is intentionally open-ended,
but should be a physical or digital work which satisfies a \textbf{need} for an intended \textbf{audience}. This could be:

\begin{itemize}
    \item \textbf{A software product.} For example a game, an app, or a reusable software component. This could be a marketable product, or a demo / prototype.
    \item \textbf{A written dissertation.} That is, a long-form piece of academic writing, based on an in-depth review of relevant academic literature and possibly a piece of primary research,
        making a contribution to knowledge in a particular specialist field.
    \item \textbf{A portfolio of creative works.} For example, a portfolio of art assets, presented professionally and coherently, demonstrating the student's artistic process.
    \item \textbf{Something else.} The brief is intentionally open-ended, and you are welcome to discuss with your supervisor what might constitute a suitable artefact.
\end{itemize}

\section*{Module overview}

For this module, you are required to complete one assignment, composed of two parts: a proposal \textbf{pitch} and an \textbf{artefact}.

Note that this module is worth 60 credits, which is twice as many credits as your previous modules.
As such the expectation is that the amount of work you put into this module is roughly double that of previous modules.

\subsubsection*{GAM705 Final Major Project: Proposal (20\%)}

Your \textbf{proposal pitch} will present your project concept and disseminate your initial research and experimentation. In delivering this proposal pitch you will evidence a familiarity with the wider context of your project, how it relates to the relevant academic literature and the value of the work to be carried out. You are also required to address the ethical issues surrounding your project, and justify your proposed research methods accordingly. A plan should be present that shows you have thought about the time constraints of the module, potential blockers that might hinder development and the milestones that will ensure project delivery. 

You will be provided with a template to help shape the proposal but you \textbf{should} adapt it to suit the requirements of your project. 

\subsubsection*{GAM705: Final Major Project: Artefact (80\%)}

The nature of the artefact will vary dramatically between individuals. However, your work will be assessed on the same shared criteria of viability, design, innovation, functional coherence, quality of defence. This set of criteria have been selected for their transferability and may have slightly different meaning depending on your specialism. For instance, software design usually refers to the planning and implementation of code. Whereas, user experience design will include considerations for the user journey and the perceived usability of an artefect. It is critically important to recognise that in this project you \textbf{must} evidence an iterative process and document your critical thinking. It is your choice how you document your process but tools such as blogs, version control, design portfolios, sketchbooks and many others can be used. When it comes to assessment, it is vital that the assessor is able to identify how your project has developed from initial concept, through a series of prototypes and arrived at the final deliverable.
