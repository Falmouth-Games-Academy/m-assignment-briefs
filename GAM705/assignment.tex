\chapter{Assignment Structure for the Major Project} \label{ch:assignment}

The Major Project is a single assignment consisting of two formally assessed parts: the proposal pitch (weighted 20\%), and a presentation of the final artefact (weighted 80\%).
This chapter breaks down the key deliverables for the project.
Recall that \textbf{formative} submissions are opportunities for feedback on works-in-progress, whereas \textbf{summative} submissions are formally assessed pieces of work which contribute to your final mark.

\section*{Part A: Project Concept}

This part consists of a \textbf{single formative submission}.

To complete this part, \textbf{conceive} of a topic for your project, by:
\begin{enumerate}[label=(\roman*)]
	\item \textbf{reviewing} the state-of-the-art from an industry and/or academic point of view; and
	\item \textbf{deriving} a key question or problem from the review to motivate your work.
\end{enumerate}

Discuss this topin with your \textbf{supervisor} in the timetabled supervision meetings in weeks 1 and 3. You will receive \textbf{informal feedback} on your proposal,
which you should use to iterate upon your ideas.

\section*{Part B: Proposal Pitch}

This part consists of a \textbf{single summative submission}, worth \textbf{20\%} of the overall marks for this assignment.
This work is \textbf{individual} and will be assessed on a \textbf{criterion} basis, according to the rubric in Chapter~\ref{ch:rubrics}.

To complete this part, \textbf{prepare} a 10-minute proposal pitch that will answer the following high-level questions:
\begin{enumerate}[label=(\roman*)]
	\item What is the context of your project? How does it fit into your specific field?
	\item What are the key results, from academic literature and/or from state-of-the-art, upon which your project will be built?
	\item What are the key research questions that you will seek to answer in your project?
	\item What will be the final format of your artefact?
	\item Who is the audience for your artefact, and what need does it fulfil for them?
	\item What are the key legal, social, ethical, and/or professional issues associated with your project?
\end{enumerate}

\textbf{Deliver} your presentation in the timetabled session in week 4, and be prepared to \textbf{defend} your proposal in Q\&A.

You will receive immediate \textbf{informal feedback} from the tutors in the session, and \textbf{formal feedback} within 15 working days of the timetabled proposal pitch session.

\section*{Part C: Project Delivery}

This part consists of \textbf{multiple formative submissions}.

To complete this part, carry out an extensive research and innovation project using industry standard project management principles and techniques. You will receive regular \textbf{informal feedback} about your work through meetings with your supervisor. Iteratively improve the artefact and show it to your supervisor in a timetabled meeting. As the requirements for the artefact will vary by project, consult with your supervisor to verify whether or not the artefact is adequate for the desired purpose.

\section*{Part D: Artefact Submission}

This part consists of a \textbf{single summative submission} that feeds into Part E.

To complete this part, upload a \texttt{.zip} file containing the final version of your artefact and any assets/dependencies to the LearningSpace.

Note that LearningSpace will only accept a single \texttt{.zip} file, with a maximum size of 1 gigabyte.
If your file is too large, you may be permitted to submit a link to materials hosted elsewhere,
e.g.\ a repository on the Games Academy Git server or a shared folder on the university's OneDrive.
Please discuss this with your supervisor and/or the module leader prior to submission.

This work is \textbf{individual} and will be assessed on a \textbf{criterion} basis, according to the rubric in Chapter~\ref{ch:rubrics}. You will receive \textbf{formal feedback} within 15 working days of the summative deadline.

\section*{Part E: Viva Presentation}

This part consists of a \textbf{single summative submission}. Together, parts D and E are worth \textbf{80\%} of the overall marks for this assignment.
This work is \textbf{individual} and will be assessed on a \textbf{threshold} basis, according to the rubric in Chapter~\ref{ch:rubrics}.

To complete this part, \textbf{prepare} a 15-minute presentation that will answer the following high-level questions:
\begin{enumerate}[label=(\roman*)]
	\item What is the context and high-level concept of your project?
	\item How did you approach the development of the project?
	\item What answers did you find to the key research questions that you set out to explore?
	\item Does your artefact fulfil the need you identified for your target audience? Why / why not?
	\item What are the wider implications and value of your work? What potential is there for future work?
\end{enumerate}

If appropriate to the project, your presentation should include a 5-minute demonstration of your artefact.
You can demonstrate the artefact ``live'' in the presentation if you wish, however it is strongly advised you also prepare a pre-recorded video demonstration as a backup.

\textbf{Deliver} your presentation in the timetabled session in week 13, and be prepared to \textbf{defend} your project in Q\&A.

You will receive immediate \textbf{informal feedback} from tutors, and \textbf{formal feedback} within 15 working days of the summative deadline.
