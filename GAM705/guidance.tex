\chapter{Guidance on Completing your Project}

\section*{Ethical Clearance}

\textbf{All} students must follow Falmouth University's Research Ethics Policy. In practice, this means you may need to complete the Research Ethics Approval Application Form and obtain ethical clearance \textbf{before} using the research artefact you have created to collect primary data. Even if collecting such data presents minimal to no risk. Furthermore, if your project involves human participants (e.g. for testing a game system you have developed), or presents a significant ethical risk (e.g., systems that process personal data) then you will need to have completed the full Research Ethics Approval Application Form available on the Falmouth Integrity \& Ethics page linked below. You \textbf{should} discuss matters of an ethical nature with your research supervisor and pay particular attention to how you address these issues in your research methodology. The ethics form should be submitted as part of the final artefact submission.

Read more about Falmouth University's Integrity and Ethics policy here: \url{https://www.falmouth.ac.uk/research/research-ethics-integrity}

\section*{Project Management}

The final major project can be quite a daunting and intimidating prospect. Many students find it challenging because of its self-directed nature and the accompanying discipline and dedication needed to follow it through. However, do not worry! Completing the project is not only intellectually liberating and a mark of your academic independence, but will provide you with a sense of achievement and satisfaction.

It will also constitute an key indicator---a symbol---of your competence to potential employers. It gives you something you can showcase in interview and discuss in considerable depth and with enthusiasm. Successful completion of such a project demands a mastery of core employability skills including: initiative; problem solving; communication, both written and spoken; self-regulated learning; as well as planning and management. Though the stretching of your ability in these areas to limits you never thought possible will likely be rather uncomfortable, you feel better for it when you graduate.

A pitfall that many students fall into, however, is time management. Minimise your procrastination and try to chip away at your work a little every day! Although your milestones will vary depending on the nature of your project, ideally the research artefact should be near its completion towards week 10 of the module. Be sure to document, organise and reflect on your process as you go. Do not underestimate the value of this process! 

It is critically important that you consider \textbf{project management} again at this stage. Consider an artefact development life-cycle that is appropriate to your project. Ensure you use appropriate project management tools including critical path analysis, Gantt charts, and burn-down charts to keep track of your progress. Also, do not underestimate the importance of the \textbf{validation and verification} aspect of the research artefact. Where applicable, you must ensure sufficient time is made to enact quality assurance practices that will defend the integrity of your research by showing that your research project was appropriately managed and your artefact was constructed through the sound application of pipelines and processes applicable to your specialism.

If you encounter any issues with respect to your time management, please consult your supervisor who can provide you with advice. They are there to support you, so take advantage of their experience. Also, ensure that you take advantage of the support services offered by the Academic Skills Team (ASK): \url{http://ask.fxplus.ac.uk}

\section*{Frequently Asked Questions}

\subsection*{Can I change my project concept partway through?} 

It is a natural part of any project that your concept and aims will shift as the project evolves, and as you receive feedback from your supervisor and from others.
The project you propose initially is not set in stone, and allowing it to evolve over time is in fact a key part of agile project management.
However, caution is advised before making radical changes, especially as the final deadline for the project draws closer.
In any case, your supervisor can advise.

\subsection*{Can I use asset packs or open-source software libraries in my project?}

Yes --- this is a good way of managing the scope of a complex project, and allowing you to work around gaps in your skill set.
However it is important to make sure that any work that is not your own is properly credited.
Also be aware that you will only receive credit for your own work, so you must ensure that you are adding significant value to any third-party work you use.

\subsection*{Is a written submission required?}

All students must submit their artefact and deliver the two presentations described in Chapter~\ref{ch:assignment}.
If the artefact is not itself a written dissertation, then no written submission is required.

\subsection*{Are slides required for the assessed presentations?}

Slides are not mandatory, however if you choose not to use them then you should give careful thought to what other visual aids might help you get your concept across to the audience.
Slides may be prepared using your choice of presentation software.

\subsection*{Where can I ask further questions?}

You may contact your supervisor or the module leader, either by email or through Microsoft Teams. Refer to LearningSpace for staff contact details.

